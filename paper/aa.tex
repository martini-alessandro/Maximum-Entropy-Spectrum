%                                                                 aa.dem
% AA vers. 8.2, LaTeX class for Astronomy & Astrophysics
% demonstration file
%                                                       (c) EDP Sciences
%-----------------------------------------------------------------------
%
%\documentclass[referee]{aa} % for a referee version
%\documentclass[onecolumn]{aa} % for a paper on 1 column  
%\documentclass[longauth]{aa} % for the long lists of affiliations 
%\documentclass[rnote]{aa} % for the research notes
%\documentclass[letter]{aa} % for the letters 
%\documentclass[bibyear]{aa} % if the references are not structured 
% according to the author-year natbib style

%
\documentclass{aa}  

%
\usepackage{graphicx}
%%%%%%%%%%%%%%%%%%%%%%%%%%%%%%%%%%%%%%%%
\usepackage{txfonts}
\usepackage{color}
\usepackage{calc}
\usepackage{amsmath,amssymb,graphicx}
\usepackage{tensor}
\usepackage{bm}
\usepackage{times}
\usepackage[varg]{txfonts}
\usepackage[colorlinks, linkcolor=blue, pdfborder={0 0 0}, breaklinks=true]{hyperref}
\usepackage{float}
\usepackage{dcolumn}
\usepackage[nolist,nohyperlinks]{acronym}
\usepackage{xspace}
\usepackage[abs]{overpic}
\usepackage{pict2e}
\usepackage{enumitem}
\usepackage[usenames,dvipsnames]{xcolor}
\usepackage[utf8]{inputenc}
\usepackage{gensymb}
\usepackage[capitalize]{cleveref}
\Crefname{figure}{Fig.}{Figs.}% {<type>}{<singular>}{<plural>}
\usepackage[normalem]{ulem}
\usepackage{mathrsfs}
\interfootnotelinepenalty=10000
\usepackage{array}
\usepackage{multirow}
\usepackage{subfigure}
\usepackage{tabularx}
\usepackage{etoolbox}
\usepackage{lineno}
\linenumbers

% WATERMARK settings
%\usepackage{draftwatermark}
%\SetWatermarkText{Draft-V2}
%\SetWatermarkScale{1.3}
%\SetWatermarkColor[gray]{0.85}
%\usepackage{draftwatermark}
%\SetWatermarkText{Draft}
%\SetWatermarkScale{5}

\usepackage{scrextend}
\usepackage{textgreek}
%\renewcommand{\mymacro}[1]{{\ensuremath{#1}}}

%\renewcommand{\ThisEvent}{GW190521\xspace}
\newcommand{\Msun}{\ensuremath{\mathrm{M}_\odot}}
\newcommand{\likelihood}{\ensuremath{\mathcal{L}}}
\newcommand{\note}[1]{\textit{\color{blue}Note: #1}}
\newcommand{\rough}[1]{\textit{\color{brown} #1}}
\newcommand{\todo}[1]{\textbf{\color{red} Todo: #1}}


%%%%%%%%%%%%%%%%%%%%%%%%%%%%%%%%%%%%%%%%
%\usepackage[options]{hyperref}
% To add links in your PDF file, use the package "hyperref"
% with options according to your LaTeX or PDFLaTeX drivers.
%
\begin{document} 


   \title{Hydrodynamics of giant planet formation}

   \subtitle{I. Overviewing the $\kappa$-mechanism}

   \author{G. Wuchterl
          \inst{1}
          \and
          C. Ptolemy\inst{2}\fnmsep\thanks{Just to show the usage
          of the elements in the author field}
          }

   \institute{Institute for Astronomy (IfA), University of Vienna,
              T\"urkenschanzstrasse 17, A-1180 Vienna\\
              \email{wuchterl@amok.ast.univie.ac.at}
         \and
             University of Alexandria, Department of Geography, ...\\
             \email{c.ptolemy@hipparch.uheaven.space}
             \thanks{The university of heaven temporarily does not
                     accept e-mails}
             }

   \date{Received September 15, 1996; accepted March 16, 1997}

% \abstract{}{}{}{}{} 
% 5 {} token are mandatory
 
  \abstract
  % context heading (optional)
  % {} leave it empty if necessary  
{Intermediate-mass black holes (IMBH), spanning the approximate mass range 
$100$--$10^5$\,\Msun, lie between black holes (BH) formed by stellar collapse
%lying between the known population of black holes (BH) formed via stellar collapse whose 
%mergers have been observed by LIGO-Virgo, 
and the super-massive BH lying at the centre of galaxies.  Mergers of IMBH binaries are 
the most intense gravitational-wave sources accessible by the Advanced detector network. 
Searches of the first two Advanced LIGO-Virgo observing runs did not yield any 
significant IMBH binary signals.  With the increased detector sensitivity of the third
observing run (O3), GW190521, a signal consistent with a binary merger of mass $\sim 150$\,\Msun\, 
provided the direct evidence of IMBH formation.  Here 
%After the first clear IMBH detection implied by the remnant of the merger event GW190521, 
we report on a dedicated search of O3 data
%from the third Advanced LIGO-Virgo observing run 
for further IMBH binary mergers, combining both modelled (template matched filter) and 
unmodelled search methods. 
We detail the most significant new candidate events: %obtained in addition to GW190521; 
\rough{these events are, however, not strongly inconsistent with instrumental noise origin
and thus do not indicate detection of further IMBH mergers.} 
%\footnote{(this statement we need to discuss with detchar)}.  
We quantify the sensitivity of the individual search methods and of the combined search
using a suite of IMBH binary signals simulated by numerical solution of general relativity, 
including the effects of spins misaligned with the binary orbital axis, and give upper 
limits on the corresponding astrophysical merger rates.  Our most stringent limit is for 
[MASSES/SPINS] at [NUMBER]\,Gpc$^{-3}$yr$^{-1}$ (90\% confidence), a factor of [NUMBER] 
more constraining than previous LIGO-Virgo limits.  We also update the estimated rate of
mergers similar to GW190521 to \todo{NUMBER}.}
  % aims heading (mandatory)

  % conclusions heading (optional), leave it empty if necessary 
   {}

   \keywords{giant planet formation --
                $\kappa$-mechanism --
                stability of gas spheres
               }

   \maketitle
%
%________________________________________________________________
%\section{Introduction}

 \input{introduction.tex}
%__________________________________________________________________
\input{data.tex}

%BOB, GV, KC, TD, DM
%\input{searches.tex}

%KC, BOB, TD, WP, PG, DM
%\input{results.tex}

%BOB, KC, TD
\input{rate.tex}

%AP
\input{conclusion.tex}

%\acknowledgments
%\input{LVCack.tex}

%\appendix
%\input{appendix.tex}
%\input{s200114f.tex}


%\section{Conclusions}


\begin{acknowledgements}

\end{acknowledgements}


%-------------------------------------------------------------------

\end{document}


